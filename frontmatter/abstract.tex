% !TEX root = Clean-Thesis.tex
%
\pdfbookmark[0]{Abstract}{Abstract}
\chapter*{Abstract}
\label{sec:abstract}
\vspace*{-10mm}

The prosodic realization of lexical stress, the phenomenon by which certain syllable(s) in a word are accentuated more than others, is an important feature of the German phonological system, but one that can pose a considerable challenge to students learning German as a foreign language (L2). This challenge is particularly daunting for native (L1) speakers of French, as lexical stress is realized quite differently (or perhaps not at all) in French word prosody.
As pronunciation training typically demands substantial individual attention from an instructor, which is not always feasible in classroom language-learning settings, Computer Assisted Pronunciation Training (CAPT) has emerged over recent decades as a promising way of using technology to deliver individualized pronunciation training in the absence of a human teacher.
This thesis investigates how CAPT can be used to help L1 French speakers learning German as L2 improve their pronunciation with regard to lexical stress prosody.

%First, i
In an effort to illuminate the nature of L1 French learners' lexical stress errors in German, the thesis 
describes the manual annotation of lexical stress errors in a learner speech corpus,
and presents an analysis of the frequency and types of errors observed.
%presents an analysis of the frequency and types of errors observed in a learner speech corpus manually annotated for lexical stress errors.
A variety of methods for automatically diagnosing such errors in learner word utterances are then explored, including novel methods for diagnosis by means of classification using supervised machine learning, as well as by means of comparison with one or more reference utterances, i.e. the same word pronounced by a native German speaker.
The ways in which diagnoses produced by these methods can be used to deliver diverse types of feedback on these errors are then explored, including types which have not previously been utilized in German CAPT. 

In its most important contribution, this thesis describes a prototype CAPT tool, \textit{de-stress},
%: German (deutsche) System for Training and Research on Errors in Second-language Stress, 
which integrates these various diagnostic and feedback methods via an easy-to-use web interface. 
%a system for use by language learners and teachers as well as researchers of language acquisition
This tool is designed not only to provide students with feedback on their lexical stress errors, but also to facilitate research on the efficacy of the various
% diagnostic and feedback methods 
 types of diagnosis and feedback
 explored in this thesis, and to enable L2 German teachers 
 %to choose from these methods 
 to create exercises best suited to their students' needs; it thus constitutes an important step towards the development of a comprehensive, intelligent CAPT system for French learners of German. 




%\vspace*{20mm}
%
%%{\usekomafont{chapter}Abstract (different language)}\label{sec:abstract-diff} \\
%%
%%\blindtext
