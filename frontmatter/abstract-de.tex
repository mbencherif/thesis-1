% !TEX root = Clean-Thesis.tex
%
%\pdfbookmark[0]{Zusammenfassung}{Zusammenfassung}
\chapter*{Zusammenfassung}
\label{sec:abstract:de}
\vspace*{-10mm}

Die prosodische Realisierung von lexikalischer Betonung, das Phänomen, bei dem eine oder mehrere bestimmte Silben eines Wortes akzentuierter produziert werden als andere, ist ein wichtiges Merkmal des phonologischen Systems des Deutschen, was allerdings eine bedeutende Herausforderung für Lerner des Deutschen als Fremdsprache (L2) darstellen kann. Diese Herausforderung ist besonders groß für Sprecher mit Französisch als Muttersprache (L1), da lexikalische Betonung anders (vielleicht überhaupt nicht) realisiert wird. Da Aussprachetraining üblicherweise viel individuelle Aufmerksamkeit für jeden Lerner erfordert, die im Fremdsprachenunterricht nicht immer zureichend gewährleistet werden kann, entwickelte sich über die letzten Jahrzehnte mit dem computergestütztem Aussprachetraining (Computer Assisted Pronunciation Training (CAPT)) eine neue Möglichkeit heraus, seine Aussprache auch ohne einen Lehrer/eine Lehrerin individuell zu trainieren. Diese Arbeit untersucht, wie CAPT-Systeme dabei behilflich sein können, die Realisierung lexikalischer Betonung von französischen Lernern des Deutschen zu verbessern. 

In dem Bemühen der Natur der Fehler von lexikalischer Betonung durch französische Muttersprachler auf den Grund zu gehen, beschreibt die Arbeit die manuelle Annotation der lexikalischen Betonungsfehler in einem Lernerkorpus und präsentiert eine Analyse der verschiedenen Fehlertypen und deren Häufigkeiten. Unterschiedliche Methoden der automatischen Diagnose dieser Fehler in Lerneräußerungen werden untersucht, einschließlich neuer Methoden, wie beispielsweise der Klassifizierung durch überwachte maschinelle Lernverfahren oder wie Vergleiche mit einer oder mehreren Vergleichsäußerungen von einem deutschen Muttersprachler. Anschließend wird untersucht, inwiefern die Diagnosen dieser Methoden genutzt werden können, um verschiedene Arten von Feedback zu präsentieren. Dies bezieht ebenfalls Feedbackmethoden mit ein, die vorher noch in keinem deutschen CAPT-Stystem verwendet wurden. 

Der wichtigste Beitrag dieser Arbeit ist die Beschreibung eines prototypischen CAPT Tools,  
%(\TODO{de-stress}), 
das die verschiedenen diagnostischen Verfahren und Feedbackmethoden über ein leicht zu nutzendes Interface miteinander vereint. Dieses System ist so konzipiert, dass es nicht nur Feedback zur Realisierung von lexikalischer Betonung gibt, sondern ebenfalls eine Plattform zur Effizienzanalyse der verschiedenen Diagnose- und Feedbacktypen darstellt. Es ermöglicht außerdem Lehrern von Deutsch als Fremdsprache, speziell auf die Bedürfnisse der Schüler abgestimmte Aufgaben zu erstellen. Aus diesem Grund trägt es zur weiteren Entwicklung von umfassenden, intelligenten CAPT-Systemen für französische Lerner des Deutschen bei. 


\begin{flushright}
\textit{Übersetzt aus dem Englischen von Jeanin Jügler und Frank Zimmerer}
\end{flushright}