% CONCLUSION AND OUTLOOK
%
%!TEX root = ../thesis-main.tex
%
%\part{Conclusion and outlook}
\chapter{Conclusion and outlook}
\label{chap:conclusion}

%\cleanchapterquote{You can’t do better design with a computer, but you can speed up your work enormously.}{Wim Crouwel}{(Graphic designer and typographer)}

\section{Thesis summary}
 \label{sec:conclusion:summary}

\section{Future work}
\label{sec:conclusion:future}


	\subsection{Coping with segmentation errors}
	\label{sec:segmentation:errors}
	\TODO{}
	
	Forced alignment is not a perfect method; because of the constraints put on the recognition system, the aligner will always find a match between the given text and audio, even if they do not correspond. Incorrect segmentation can lead to mistakes in diagnosis, so CAPT systems must have a means of reducing, or at least monitoring, the amount of error introduced by inaccurate segmentation \citep{Eskenazi2009}. 	
	In the proposed CAPT tool, this function may be served by the development of a simple sentence- and/or word-level confidence measure. 
	While it is very difficult to compute such a measure directly from the decoding scores of the forced aligner, it may be possible to determine from the aforementioned accuracy evaluation which types of boundaries (e.g. between a sonorant and a vowel) the aligner typically has trouble detecting accurately, and then to calculate, for a given utterance, the proportion of error-prone boundaries. While a very simplistic measure, this could nevertheless provide some indication of when (not) to trust the automatic alignment, thus impacting decisions on how and whether to attempt error diagnosis (or feedback).
	% based on the boundary error rates found in \cref{sec:segmentation:eval}. 
	Other error-management strategies may also be explored, such as the type of error-filtering methods described by \textcite{Mesbahi2011,Bonneau2012,Orosanu2012}, in which utterances which do not correspond to the expected text are detected and rejected before alignment is attempted.
	