\documentclass[a4paper]{article}

\usepackage{INTERSPEECH2015}

\usepackage{graphicx}
\usepackage{amssymb,amsmath,bm}
\usepackage{textcomp}
\usepackage{cleveref}

\def\vec#1{\ensuremath{\bm{{#1}}}}
\def\mat#1{\vec{#1}}

\RequirePackage[dvipsnames]{xcolor}
\newcommand{\TODO}[1]{{\color{red}\textbf{[TODO #1]}}}



\sloppy % better line breaks
\ninept


\title{Automatic classification of lexical stress errors for German CAPT}

%%%%%%%%%%%%%%%%%%%%%%%%%%%%%%%%%%%%%%%%%%%%%%%%%%%%%%%%%%%%%%%%%%%%%%%%%%
%% If multiple authors, uncomment and edit the lines shown below.       %%
%% Note that each line must be emphasized {\em } by itself.             %%
%% (by Stephen Martucci, author of spconf.sty).                         %%
%%%%%%%%%%%%%%%%%%%%%%%%%%%%%%%%%%%%%%%%%%%%%%%%%%%%%%%%%%%%%%%%%%%%%%%%%%
%\makeatletter
%\def\name#1{\gdef\@name{#1\\}}
%\makeatother
%\name{{\em Firstname1 Lastname1, Firstname2 Lastname2, Firstname3 Lastname3,}\\
%      {\em Firstname4 Lastname4, Firstname5 Lastname5, Firstname6 Lastname6,
%      Firstname7 Lastname7}}
%%%%%%%%%%%%%%% End of required multiple authors changes %%%%%%%%%%%%%%%%%

\makeatletter
\def\name#1{\gdef\@name{#1\\}}
\makeatother \name{{\em%
  Anjana Sofia Vakil, J\"{u}rgen Trouvain
  %Author Name$^1$, Co-author Name$^2$
  }}

\address{%
  %$^1$Author Affiliation \\
  %$^2$Co-author Affiliation \\
  Department of Computational Linguistics \& Phonetics\\
  Saarland University, Saarbr\"{u}cken, Germany\\
  {\small \tt [anjanav,trouvain]@coli.uni-saarland.de}
}

%\twoauthors{Karen Sp\"{a}rck Jones.}{Department of Speech and Hearing \\
%  Brittania University, Ambridge, Voiceland \\
%  {\small \tt Karen@sh.brittania.edu} }
%  {Rose Tyler}{Department of Linguistics \\
%  University of Speechcity, Speechland \\
%  {\small \tt RTyler@ling.speech.edu} }

%
\begin{document}



  \maketitle
  %
  \begin{abstract}
  
    (200 word limit) 
  
    
    
    
  \end{abstract}
  %
  \noindent{\bf Index Terms}: computer-assisted pronunciation training, CAPT, word prosody, German


  \section{Introduction}
  
  \cite{Vakil2015}
  
  For adult learners of a second language (L2), the phonological system of the L2 can pose a variety of difficulties. For certain L2s, such as German or English, one important difficulty involves the accurate prosodic realization of lexical stress, i.e. the accentuation of certain syllable(s) in a given word, with the placement of stress within a word varying freely and carrying a contrastive function in such languages \TODO{cite}. Lexical stress is an important part of German word prosody, and has been found to have an impact on the intelligibility of non-native German speech \TODO{cite}. Coping with this phenomenon in German is especially challenging for native (L1) French speakers, because lexical stress is realized very differently (or perhaps not at all) in the French language \TODO{cite}.
  
  To overcome this difficulty and improve their L2 word prosody, learners typically need to have their pronunciation errors pointed out and corrected by a language instructor; unfortunately, the lack of attention typically given to pronunciation in the foreign language classroom, coupled with the historic lack of instructor training in phonetics/phonology, make this level of individualized attention not always feasible in a classroom setting \TODO{cite}. Fortunately, advances in Computer-Assisted Pronunciation Training (CAPT) over recent decades have made it possible to automatically provide highly individualized analysis of learners' prosodic errors, as well as feedback on how to correct them, and thus to help learners achieve more intelligible pronunciation in the target language (Witt, 2012). However, while much research has gone into the creation and improvement of CAPT systems for English, relatively little work has been done on the development of CAPT systems for German, and even less on the treatment of prosodic errors in such systems \TODO{cite (Hirschfeld?)}.
  
  This paper describes work that advances the state of German CAPT by applying machine learning methods to the task of diagnosing lexical stress errors in non-native German speech, a necessary prerequisite for delivering individualized corrective feedback on such errors in a CAPT system. The paper is organized as follows: \Cref{sec:bkgd} provides background on the phenomenon of lexical stress as it is realized in German and French word prosody, motivates the creation of CAPT systems that address this error specifically, and summarizes some past work related to this topic. \Cref{sec:data} describes the manual annotation of lexical stress errors in a small corpus of L2 German speech, carried out to create labeled training and test data for the classification experiments explained in \cref{sec:method}. \Cref{sec:results} presents and analyzes the results of these experiments. Finally, \cref{sec:conc} offers some concluding remarks and outlines possible directions for future extensions of this work.
  
  %Lexical stress, the phenomenon by which a given syllable is accorded a higher level of prominence than other syllables in a given word, serves a contrastive function in some languages (e.g. German) but not others (e.g. French). Lexical stress is very important in German, and may have a large impact on the intelligibility of L2 German speech (see Section 2.4.1); however, given that lexical stress is realized extremely differently (or not at all) in French (see Section 2.3.2), the correct prosodic realization of lexical stress in German is notoriously difficult for L1 French speakers (see Section 2.3.3).
  %
%Computer-Assisted Pronunciation Training (CAPT) systems have the potential to automati- cally provide highly individualized analysis of such learner errors, as well as feedback on how to correct them, and thus to help learners achieve more intelligible pronunciation in the target language (Witt, 2012). The thesis project described here aims to advance German CAPT by creating a tool which will diagnose and offer feedback on lexical stress errors in the L2 German speech of L1 French speakers, in the hopes of ultimately helping these learners become more intelligible when speaking German.
  


	\section{Background and related work}
	\label{sec:bkgd}
	
	\section{Data}
	\label{sec:data}
		\subsection{The IFCASL corpus}
		\subsection{Annotation of lexical stress realizations}
		\subsection{Inter-annotator agreement}
		\subsection{Error distribution}
	
	\section{Evaluation method}
	\label{sec:method}
	    \subsection{Feature sets}
	    \subsection{Datasets for training and testing}

	\section{Results}
	\label{sec:results}
		\subsection{Feature performance}
		\subsection{Performance on unknown words}
		\subsection{Performance on unknown speakers}
	
	%\section{Alternative error diagnosis methods}


	%\section{Discussion}

	\section{Conclusions and future work}
	\label{sec:conc}
	
	%\section{Future work}
	%\label{sec:future}



  %TODO \section{Acknowledgements}
  



  \newpage
  \eightpt
  \bibliographystyle{IEEEtran}

  \bibliography{../../library.bib}

%  \begin{thebibliography}{9}
%    \bibitem[1]{Davis80-COP}
%      S.\ B.\ Davis and P.\ Mermelstein,
%      ``Comparison of parametric representation for monosyllabic word recognition in continuously spoken sentences,''
%      \textit{IEEE Transactions on Acoustics, Speech and Signal Processing}, vol.~28, no.~4, pp.~357--366, 1980.
%    \bibitem[2]{Rabiner89-ATO}
%      L.\ R.\ Rabiner,
%      ``A tutorial on hidden Markov models and selected applications in speech recognition,''
%      \textit{Proceedings of the IEEE}, vol.~77, no.~2, pp.~257-286, 1989.
%    \bibitem[3]{Hastie09-TEO}
%      T.\ Hastie, R.\ Tibshirani, and J.\ Friedman,
%      \textit{The Elements of Statistical Learning -- Data Mining, Inference, and Prediction}.
%      New York: Springer, 2009.
%    \bibitem[4]{YourName15-XXX}
%      F.\ Lastname1, F.\ Lastname2, and F.\ Lastname3,
%      ``Title of your INTERSPEECH 2015 publication,''
%      in \textit{Interspeech 2015 -- 16\textsuperscript{th} Annual Conference of the International Speech Communication Association, September 06--10, Dresden, Germany, Proceedings}, 2015, pp.~100--104.
%  \end{thebibliography}

\end{document}
