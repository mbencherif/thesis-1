\documentclass[a4paper]{article}

\usepackage{INTERSPEECH2015}

\usepackage{graphicx}
\usepackage{amssymb,amsmath,bm}
\usepackage{textcomp}
\usepackage{cleveref}

\def\vec#1{\ensuremath{\bm{{#1}}}}
\def\mat#1{\vec{#1}}

\usepackage[dvipsnames]{xcolor}
\newcommand{\TODO}[1]{{\color{red}\textbf{[TODO #1]}}}

\usepackage{enumitem}



\sloppy % better line breaks
\ninept


\title{Automatic classification of lexical stress errors for German CAPT}

%%%%%%%%%%%%%%%%%%%%%%%%%%%%%%%%%%%%%%%%%%%%%%%%%%%%%%%%%%%%%%%%%%%%%%%%%%
%% If multiple authors, uncomment and edit the lines shown below.       %%
%% Note that each line must be emphasized {\em } by itself.             %%
%% (by Stephen Martucci, author of spconf.sty).                         %%
%%%%%%%%%%%%%%%%%%%%%%%%%%%%%%%%%%%%%%%%%%%%%%%%%%%%%%%%%%%%%%%%%%%%%%%%%%
%\makeatletter
%\def\name#1{\gdef\@name{#1\\}}
%\makeatother
%\name{{\em Firstname1 Lastname1, Firstname2 Lastname2, Firstname3 Lastname3,}\\
%      {\em Firstname4 Lastname4, Firstname5 Lastname5, Firstname6 Lastname6,
%      Firstname7 Lastname7}}
%%%%%%%%%%%%%%% End of required multiple authors changes %%%%%%%%%%%%%%%%%

\makeatletter
\def\name#1{\gdef\@name{#1\\}}
\makeatother \name{{\em%
  Anjana Sofia Vakil%, J\"{u}rgen Trouvain
  %Author Name$^1$, Co-author Name$^2$
  }}

\address{%
  %$^1$Author Affiliation \\
  %$^2$Co-author Affiliation \\
  Department of Computational Linguistics \& Phonetics\\
  Saarland University, Saarbr\"{u}cken, Germany\\
  {\small \tt 
  %[%
  anjanav%
  %,trouvain%
  %]%
  @coli.uni-saarland.de}
}

%\twoauthors{Karen Sp\"{a}rck Jones.}{Department of Speech and Hearing \\
%  Brittania University, Ambridge, Voiceland \\
%  {\small \tt Karen@sh.brittania.edu} }
%  {Rose Tyler}{Department of Linguistics \\
%  University of Speechcity, Speechland \\
%  {\small \tt RTyler@ling.speech.edu} }

%
\begin{document}



  \maketitle
  %
  \begin{abstract}
  
    \TODO{Abstract (200 word limit)}
  
  	This very useful filler sentence has exactly ten different words. This very useful filler sentence has exactly ten different words. This very useful filler sentence has exactly ten different words. This very useful filler sentence has exactly ten different words. This very useful filler sentence has exactly ten different words. This very useful filler sentence has exactly ten different words. This very useful filler sentence has exactly ten different words. This very useful filler sentence has exactly ten different words. This very useful filler sentence has exactly ten different words. This very useful filler sentence has exactly ten different words. This very useful filler sentence has exactly ten different words. This very useful filler sentence has exactly ten different words. This very useful filler sentence has exactly ten different words. This very useful filler sentence has exactly ten different words. This very useful filler sentence has exactly ten different words. This very useful filler sentence has exactly ten different words. This very useful filler sentence has exactly ten different words. This very useful filler sentence has exactly ten different words. This very useful filler sentence has exactly ten different words. This very useful filler sentence has exactly ten different words. 
    
    
    
  \end{abstract}
  %
  \noindent{\bf Index Terms}: computer-assisted pronunciation training, CAPT, word prosody, German \TODO{are these OK?}


  \section{Introduction}
  
  %\cite{Vakil2015}
  
    
  %Lexical stress, the phenomenon by which a given syllable is accorded a higher level of prominence than other syllables in a given word, serves a contrastive function in some languages (e.g. German) but not others (e.g. French). Lexical stress is very important in German, and may have a large impact on the intelligibility of L2 German speech (see Section 2.4.1); however, given that lexical stress is realized extremely differently (or not at all) in French (see Section 2.3.2), the correct prosodic realization of lexical stress in German is notoriously difficult for L1 French speakers (see Section 2.3.3).
  %
%Computer-Assisted Pronunciation Training (CAPT) systems have the potential to automati- cally provide highly individualized analysis of such learner errors, as well as feedback on how to correct them, and thus to help learners achieve more intelligible pronunciation in the target language (Witt, 2012). The thesis project described here aims to advance German CAPT by creating a tool which will diagnose and offer feedback on lexical stress errors in the L2 German speech of L1 French speakers, in the hopes of ultimately helping these learners become more intelligible when speaking German.
  
  
  For adult learners of a second language (L2), the phonological system of the L2 can pose a variety of difficulties. For certain L2s, such as German or English, one important difficulty involves the accurate prosodic realization of lexical stress, i.e. the accentuation of certain syllable(s) in a given word, with the placement of stress within a word varying freely and carrying a contrastive function in such languages \cite{Cutler2005}. Lexical stress is an important part of German word prosody, and has been found to have an impact on the intelligibility of non-native German speech \cite{Hirschfeld1994}. Coping with this phenomenon in German is especially challenging for native (L1) French speakers, because lexical stress is realized very differently (or perhaps not at all) in the French language \cite{Dupoux2008,Michaux2013}.
  
  To overcome this difficulty and improve their L2 word prosody, learners typically need to have their pronunciation errors pointed out and corrected by a language instructor; unfortunately, the lack of attention typically given to pronunciation in the foreign language classroom, 
  along with other factors such as high student-to-teacher ratios,
  %coupled with the historic lack of instructor training in phonetics/phonology, 
  make this level of individualized attention not always feasible in a classroom setting \cite{Neri2002,Derwing2005,Hirschfeld2007}. Fortunately, advances in Computer-Assisted Pronunciation Training (CAPT) over recent decades have made it possible to automatically provide highly individualized analysis of learners' prosodic errors, as well as feedback on how to correct them, and thus to help learners achieve more intelligible pronunciation in the target language. However, while much research has gone into the creation and improvement of CAPT systems for English (see e.g. \cite{Eskenazi2009,Witt2012}), relatively little work has been done on the development of CAPT systems for German, especially on those targeting errors in German prosody.
  
  This paper describes work that advances the state of German CAPT by applying machine learning methods to the task of diagnosing lexical stress errors in non-native German speech, a necessary prerequisite for delivering individualized corrective feedback on such errors in a CAPT system. The paper is organized as follows: \Cref{sec:bkgd} provides background on the phenomenon of lexical stress as it is realized in German and French word prosody, motivates the creation of CAPT systems that address this error specifically, and summarizes some past work related to this topic. \Cref{sec:data} describes the manual annotation of lexical stress errors in a small corpus of L2 German speech, carried out to create labeled training and test data for the classification experiments explained in \cref{sec:method}. \Cref{sec:results} presents and analyzes the results of these experiments. Finally, \cref{sec:conc} offers some concluding remarks and outlines possible directions for future work.



	\section{Background and related work}
	\label{sec:bkgd}
	
%When there is a typological difference between some segmental or prosodic feature(s) of a language learner’s L1 compared to the target L2, there is a particular need for pronunciation training to bridge this gap. In the case of the French-German language pair, the prosodic realization of lexical stress is one feature which marks a striking difference between the languages.

Broadly speaking, lexical stress is the phenomenon of how a given syllable is accentuated within a word \cite{Cutler2005}, i.e. how a syllable is given a more prominent role such that this syllable is perceived as ``standing out'' \cite{Dogil1999}. This perceived prominence of a syllable is a function not merely of the segmental characteristics of the uttered syllable, i.e. the speech sounds it contains, but rather of its (relative) suprasegmental properties, namely:
\begin{itemize}[topsep=.5em,noitemsep]
\item duration, which equates on the perceptual level to length;
\item fundamental frequency (F0), which corresponds to perceived pitch; and 
\item intensity (energy or amplitude), which perceptually equates to loudness.
\end{itemize}

%As Cutler (2005) points out, different languages make use of this suprasegmental information in different ways. 
In variable-stress languages, such as German and English, 
the location of lexical stress in a word is not always predictable,
%it is not always possible to predict which syllable in a word will carry the stress, 
and therefore knowing a word requires, in part, knowing its stress pattern. This allows lexical stress to serve a contrastive function in these languages, 
%such that two words may share exactly the same sequence of phones and nevertheless be distinguished exclusively by their stress pattern, as is the case with 
e.g. distinguishing \textit{UMfahren} (to drive around) from \textit{umFAHRen} (to run over with a car) in German. 
%Because stress carries meaning thus, native speakers of such languages are sensitive to stress patterns, and readily able to perceive differences in stress. 
Furthermore, in German, misplaced stress can disrupt understanding even in cases where there is no stress-based minimal pair \cite{Hirschfeld1994}.
%, supporting the theory that speakers of free-stress languages rely to a large extent on stress information in the recognition of spoken words (Cutler, 2005).
%
However, in fixed-stress languages, stress is completely predictable, as it always falls on a certain position in the word (e.g. the final syllable), making the lexical stress pattern less crucial to the knowledge of a word than in variable-stress languages. 
%Lexical stress may not be as crucial to the knowledge of a word in these languages as in the variable-stress languages. 
Furthermore, 
%although lexical stress is realized in these languages, 
in fixed-stress languages there may be a weaker distinction between stressed and unstressed syllables.
While French has often been categorized as a fixed-stress language, given that word-final syllables are given prominence when a French word is pronounced in isolation, some argue that it may be more properly considered a language without lexical stress, in that speakers do not seem to accentuate any syllable within the word, with word-final lengthening effects explained by interactions with the realization of phrasal accent (lengthening of the final syllable in each prosodic group or phrase) \cite{Dupoux2008,Michaux2013}. 
Regardless, French has no contrastive word-level stress \cite[p.~89]{Michaux2013}, 
and in this respect differs considerably from German.
%which constitutes a significant difference between this language and a language with variable, contrastive lexical stress such as German or English.

This difference between the languages leads us to expect French learners of German to have difficulties with both perception and production of lexical stress prosody.
Although little research has been done on the nature of lexical stress errors for this particular L1-L2 pair, Hirschfeld and Trouvain \cite{Hirschfeld2007} report that such errors are commonly observed in German spoken by French natives.
Research on French speakers' perception of Spanish, another contrastive-stress language, has revealed that these speakers seem to be ``deaf'' to lexical stress, i.e. seem to have significant and lasting difficulty perceiving and remembering stress contrasts \cite{Dupoux2008}. 
With respect to production, studies of L2 Dutch have shown that French speakers, especially beginners, make systematic errors with lexical stress, exhibiting a tendency to stress the final syllable of Dutch words even when stress should be placed on the initial or medial syllable \cite{Michaux2012,Michaux2013}. Similar findings have also been reported for French learners of English \cite{Bonneau2011}.
%As a result of this difference in the sound systems of the two languages, native speakers of French may generally be expected to lack the sensitivity to stress patterns possessed by native speakers of German. Indeed, this has been borne out by research by Dupoux et al. (2008), who found that native French speakers seem to be “deaf” to lexical stress, insofar as they have significant and lasting difficulty perceiving lexical stress in Spanish, another language in which stress serves a contrastive function. This difficulty should also exist for French speakers when they are presented with German words in which the stress pattern is crucial to the word’s meaning, as in the minimal pair above.
%In addition to these difficulties with lexical stress perception, French learners of variable- stress languages such as English, German and Dutch have also been shown to have difficulties in producing stress patterns correctly. Research by Michaux et al. (2012; 2013) revealed that, as might be expected given the tendency for word- and/or phrase-final syllable prominence in French just discussed, French learners of Dutch showed a tendency to stress the final syllables of Dutch words, even when not called for by the canonical stress pattern. Indeed, with regard to German specifically, Hirschfeld and Trouvain (2007) report that lexical stress errors are commonly observed among L2 speakers with French as L1.
%In short, based on prior work on French learners of variable-stress languages, it can reason- ably be expected that L1 French learners of German as L2 will face challenges with both the perception and production of lexical stress, and that the (lack of) lexical stress system in their native language will influence their realization of lexical stress patterns in German words.	
%
	The high (anticipated) frequency of lexical stress errors in the speech of this L1-L2 group is thus one motivating factor for the creation of CAPT systems to help learners identify and correct such errors. 
	
	
	
	Another motivation behind this work's focus on lexical stress errors is the high impact such errors may have on the intelligibility of L2 German speech.
	Intelligibility, as opposed to lack of a “foreign accent,” is generally considered to be the most important goal of pronunciation training \cite{Munro1999,Neri2002,Derwing2005,Field2005,Witt2012}. The exact definition of \textit{intelligibility} is a topic of debate,
	% in the literature, as is the question of whether and how it differs from the notion of comprehensibility; here, let us follow
	but here we will follow Munro and Derwing \cite[p.~289]{Munro1999} in understanding it broadly as ``the extent to which a speaker’s message is actually understood by a listener.''
	Generally speaking, prosodic errors have often been found to have a larger impact on the perceived intelligibility of L2 speakers than segmental errors (Derwing and Munro, 2005; Hahn, 2004; Witt, 2012), and several studies have found lexical stress errors to have a particularly strong impact on intelligibility in free-stress languages like English and Dutch \cite{Cutler2005,Field2005} 
	%and German \cite{Hirschfeld1994}, 
	%Dutch, and German (Cutler, 2005; Field, 2005; Hirschfeld, 1994). Indeed, studies on perception of German L2 speech have found that among a variety of pronunciation error types, lexical stress errors have one of the most drastic impacts on intelligibility (Hirschfeld, 1994). 
	Though relatively little research has been done on how various pronunciation errors affect intelligibility in L2 German specifically, some studies suggest that 
	%prosodic errors, and 
	lexical stress errors 
	%especially, 
	may hinder intelligibility of L2 German speech more than other types of errors \cite{Hirschfeld1994,Hirschfeld2007}.
	Stress errors may also affect perception of segmental errors in the L2 learner’s speech; for example, segmental errors occurring in stressed syllables may be more noticeable than those in unstressed syllables \cite{Cutler2005,Michaux2012}. 
	%Additionally, some research indicates that prosodic errors such as lexical stress errors may have more of an impact on perceived foreign accent than segmental errors (Hahn, 2004; Witt, 2012); though it must again be stressed that intelligibility is a more important goal than lack of a foreign accent, insofar as perceived accent may contribute to difficulties being understood by native speakers, this relationship between prosody and accentedness also deserves mentioning.
	It would therefore seem that there may be a strong connection between lexical stress errors and intelligibility in L2 German speech, though more research is needed to clarify the nature of this relationship. \TODO{remove that sentence?}
	
	Though the frequency and impact of lexical stress errors in the speech of French learners of German thus constitute strong reasons to develop CAPT tools to treat such errors, in order for such systems to be viable, the feasibility of reliable automatic detection of this type of error must be demonstrated. 
	\TODO{Make rest of this par about how comparison-based diagnosis is usually used, then start new par with following sentence?}
	To our knowledge, no work has been reported on automatic classification-based diagnosis of lexical stress errors in L2 German speech, 
	%yet related work on other target languages seems encouraging. 
	but in recent years machine learning methods have been applied with apparent success to the classification of lexical stress patterns in English words. 
	Kim and Beutnagel \cite{Kim2011} experimented with various classifiers to identify stress patterns in high-quality recordings of 3- and 4-syllable English words, reporting accuracy in the 80-90\% range; in pilot experiments with low-quality recordings, however, the authors report lower accuracy: 70-80\% on L1 speech and 50-60\% on utterances by L2 speakers. 
	Similarly, Shahin et al. \cite{Shahin2012} trained Neural Networks to classify stress patterns in bisyllabic words uttered by L1 English children, 
	%with the intended application of treating childhood dysprosody, 
	and reported classification accuracy over 90\% for some stress patterns; though this work was conducted with a view to treating childhood L1 dysprosody, its relevance to our intended application of L2 CAPT is nonetheless clear.
	%
	Building on these related investigations, this paper seeks to further explore the viability of automatic classification-based detection of lexical stress errors, with a particular focus on those made by French speakers of German.
	%
	To this end, a small corpus of learner speech was manually annotated for lexical stress errors, as described in \cref{sec:data}.
	Using the resulting labeled L2 data, in addition to data from L1 German speakers, a series of supervised machine learners were trained using a variety of representations of the prosodic and other features of each word utterance (see \cref{sec:method:featuresets}, and these classifiers were evaluated with reference to the manually-produced labels of held-out test data (see \cref{sec:method:datasets}). \Cref{sec:results} presents and analyzes the results of these evaluations.
	
	
	\section{Data}
	\label{sec:data}
	
	Error-annotated speech data from German learners is a prerequisite for the supervised training and evaluation of classifiers for  lexical stress realizations in L2 German speech, yet to our knowledge no corpus of learner German with such annotation is publicly available. To fill this need, as well as to shed light on the perception of lexical stress errors in L2 German speech, a small corpus of speech by L1 French learners of German was manually annotated for such errors by native and non-native German speakers with varying levels of phonetics/phonology expertise. This section describes the data selected for annotation (\cref{sec:data:corpus}) and the method by which lexical stress realizations in this data were annotated (\cref{sec:data:annotation}), and presents an analysis of the observed inter-annotator agreement (\cref{sec:data:agreement}) and distribution of errors (\cref{sec:data:errors}) in the annotated dataset.
	
		\subsection{The IFCASL corpus of learner speech}
		\label{sec:data:corpus}		
		
		\subsection{Annotation method}
		\label{sec:data:annotation}		
		
		\subsection{Inter-annotator agreement}
		\label{sec:data:agreement}		
		
		\subsection{Error distribution}
		\label{sec:data:errors}
	
	
	
	
	\section{Method}
	\label{sec:method}
	
	    \subsection{Feature sets}
	    \label{sec:method:featuresets}
	    
	    \subsection{Datasets for training and testing}
	    \label{sec:method:datasets}

	\section{Results}
	\label{sec:results}
		\subsection{Feature performance}
		\subsection{Performance on unknown words}
		\subsection{Performance on unknown speakers}
	
	%\section{Alternative error diagnosis methods}


	%\section{Discussion}

	\section{Conclusions and future work}
	\label{sec:conc}
	
	%\section{Future work}
	%\label{sec:future}



  %TODO \section{Acknowledgements}
  



  \newpage
  \eightpt
  \bibliographystyle{IEEEtran-nourl}

  \bibliography{../../library.bib}
  %\bibliography{slaterefs.bib}

%  \begin{thebibliography}{9}
%    \bibitem[1]{Davis80-COP}
%      S.\ B.\ Davis and P.\ Mermelstein,
%      ``Comparison of parametric representation for monosyllabic word recognition in continuously spoken sentences,''
%      \textit{IEEE Transactions on Acoustics, Speech and Signal Processing}, vol.~28, no.~4, pp.~357--366, 1980.
%    \bibitem[2]{Rabiner89-ATO}
%      L.\ R.\ Rabiner,
%      ``A tutorial on hidden Markov models and selected applications in speech recognition,''
%      \textit{Proceedings of the IEEE}, vol.~77, no.~2, pp.~257-286, 1989.
%    \bibitem[3]{Hastie09-TEO}
%      T.\ Hastie, R.\ Tibshirani, and J.\ Friedman,
%      \textit{The Elements of Statistical Learning -- Data Mining, Inference, and Prediction}.
%      New York: Springer, 2009.
%    \bibitem[4]{YourName15-XXX}
%      F.\ Lastname1, F.\ Lastname2, and F.\ Lastname3,
%      ``Title of your INTERSPEECH 2015 publication,''
%      in \textit{Interspeech 2015 -- 16\textsuperscript{th} Annual Conference of the International Speech Communication Association, September 06--10, Dresden, Germany, Proceedings}, 2015, pp.~100--104.
%  \end{thebibliography}

\end{document}
