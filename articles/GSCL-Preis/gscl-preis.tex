\documentclass[11pt,a4paper,onecolumn]{article}
\usepackage[utf8]{inputenc}
\usepackage{amsmath}
\usepackage{amsfonts}
\usepackage{amssymb}
\usepackage{graphicx}
\usepackage[dvipsnames]{xcolor}
\usepackage{cleveref}


\newcommand{\TODO}[1]{{\color{red}\textbf{[TODO #1]}}}

\DeclareGraphicsExtensions{.pdf,.png}

\usepackage[							% use biblatex for bibliography
	backend=bibtex,		% 	- use biber backend (bibtex replacement) or bibtex
	bibencoding=utf8,			% 	- use auto file encode
	style=authoryear-icomp,				% 	- use alphabetic (or numeric) bib style
	natbib=true,					% 	- allow natbib commands
	hyperref=true,					% 	- activate hyperref support
	backref=true,					% 	- activate backrefs		%TODO true in final
	isbn=false,						% 	- don't show isbn tags
	url=false,							% 	- don't show url tags
	doi=false,						% 	- don't show doi tags
	urldate=long,					% 	- display type for dates
	dashed=false,					%  - don't convert duplicate author names to dashes
	maxbibnames=100,%
	maxcitenames=2,%
]{biblatex}
	
\addbibresource{../../library.bib}
\addbibresource{../../Zimmerer2015.bib}

\title{Automatic diagnosis and feedback for lexical stress errors in non-native speech: Towards a CAPT system for French learners of German}

\begin{document}
\maketitle

\section*{Abstract}

This is some text. 
\section{Introduction}

\section{Related work}

\section{Lexical stress errors by French learners of German}

In an effort to shed light on the nature of lexical stress errors in the speech of L1 French learners of German as L2, \TODO{this chapter has described} original efforts to 
	annotate and analyze such errors in a small corpus of learner speech.
	
	\TODO{As described in \cref{sec:lexstress:data,sec:lexstress:annotators,sec:lexstress:method},}
	lexical stress realizations in utterances of bisyllabic, initial-stress words (\cref{sec:lexstress:data})
	taken from the IFCASL corpus (\cite{Trouvain2013,Fauth2014}; \TODO{see also \cref{sec:intro:ifcasl}) }
	were evaluated by multiple annotators from different L1 and phonetic training backgrounds \TODO{(\cref{sec:lexstress:annotators})}.
	Annotators were asked to use a graphical annotation tool to label each recorded word utterance
	as correctly or incorrectly realizing lexical stress (i.e. the speaker clearly stressed the correct or incorrect syllable), failing to clearly realize stress (i.e. the speaker did not seem to stress either syllable), or having technical or other problems which prevented the assessment of lexical stress \TODO{(\cref{sec:lexstress:method})}.
	
	Analysis of the labels assigned by different annotators to the same utterances (\cref{sec:lexstress:agreement}) revealed that inter-annotator agreement was \TODO{NUMBERS 
	%relatively low, with 
	only ``fair'' %agreement 
	\citep{Landis1977},} on average, between each pair of annotators who labeled the same utterances. 
	Considerable variation was observed among individual annotators \TODO{(\cref{sec:agreement:overall})}, which did not seem to be explained by differences 
	in their L1 \TODO{(\cref{sec:agreement:native})} or level of 
	phonetics/phonology expertise \TODO{(\cref{sec:agreement:expert})}. 
	However, it was observed that
	L2 German speakers annotated a higher proportion of utterances as having unclear stress compared to L1 speakers \TODO{(\cref{sec:agreement:native})}, and that expert annotators judged substantially higher proportion of utterances as correctly realizing stress compared to intermediate or novice annotators \TODO{(\cref{sec:agreement:expert})}.
	\TODO{As described in \cref{sec:agreement:overall},} there also seemed to be variability in inter-annotator 
	agreement with respect to the different word types %represented 
	in the dataset, and further work is needed to discern the factors responsible for this observation (see \cref{sec:conclusion:future}). 
	
	The multiple, often conflicting, error annotations from different annotators were consolidated into a single gold-standard annotation for each utterance in the dataset \TODO{(see \cref{sec:agreement:gold})}, which served as the basis for an analysis of the frequency and type of errors produced by learners \TODO{(\cref{sec:lexstress:results})}. 
	%
	On average, approximately two-thirds of learners' utterances were deemed to realize lexical stress correctly, confirming the expectation that French learners of German frequently make errors with respect to lexical stress \TODO{(see \cref{sec:stress:expected,sec:targeting:frequency})}.
	%
	The observed frequency of such errors was considerably lower in the speech of advanced learners than that of beginners \TODO{(\cref{sec:results:level})}, and children seemed to make more errors than adult beginners \TODO{(\cref{sec:results:agegender})}; no substantial difference was observed between speakers of different genders. 
	As in the case of inter-annotator agreement, considerable variation was observed in the frequency of errors in utterances of different word types \TODO{(\cref{sec:results:wordtype})}, though once again the factors underlying this variability are not immediately evident and should be investigated in future work \TODO{(see \cref{sec:conclusion:future})}.
	
	The error annotation and analysis described in this chapter thus contribute considerably to our understanding of the difficulties L1 French speakers may have realizing lexical stress in 
	German, 
	and the fact that learners, especially beginners and children, seem to struggle with lexical stress production justifies the selection of lexical stress errors as the focus of this thesis project. Additionally, the analysis of inter-annotator agreement presented in this chapter, specifically the finding that the observed agreement was generally rather low,
	constitutes an important discovery with respect to the task of identifying such errors in learner speech:
	though further research is needed to determine why and under which conditions this is the case (see \cref{sec:conclusion:future}), it would seem that diagnosing lexical stress errors may be a challenging task for at least some L1 and L2 German speakers. If true, this has important implications for the development and evaluation of automatic error diagnosis systems, 
	which
	\TODO{will be discussed further in \cref{sec:diag:classification}.}

\section{Diagnosis of lexical stress errors}

\section{Feedback on lexical stress errors}


\printbibliography[title={References}]

\end{document}